\newpage % strana pre zadanie
\thispagestyle{empty}
\begingroup
\begin{LARGE}
	Namiesto tejto strany treba vložiť zadanie záverečnej práce.
	Do elektronickej verzie práce vložte naskenované zadanie záverečnej práce ako obrázok zväčšený na celú veľkosť papiera
\end{LARGE}
\endgroup

\newpage % čestné vyhlásenie
\thispagestyle{empty}
\mbox{}
\vfill
\section*{Čestné vyhlásenie}
Vyhlasujem, že som zadanú diplomovú prácu vypracoval samostatne, pod odborným vedením vedúceho práce/školiteľa a používal som len literatúru uvedenú v práci. \\
\vspace{1cm}
Žilina 3. marca 2022 \\
\vspace{1cm}	
\null\hfill podpis

\newpage % poďakovanie
\thispagestyle{empty}
\mbox{}
\vfill
\section*{Poďakovanie}
(Poďakovanie nie je povinná časť záverečnej práce)

\newpage % abstrakt
\thispagestyle{empty}
\section*{Abstrakt}

\hspace{1.25cm}Abstrakt býva spravidla informatívny a zachováva tematické a štýlové vlastnosti primárneho dokumentu. Podľa možnosti obsahuje kvalitatívnu a kvantitatívnu informáciu obsiahnutú v dokumente. Abstrakt sa píše súvisle ako jeden odsek a jeho rozsah je spravidla 100 až 500 slov. Súčasťou abstraktu je 3 až 5 kľúčových slov (Metodické usmernenie 14/2009-R, 2009; Katuščák, 2004).
Príklad: \\
SITKOVÁ, Zuzana: Hydrochemické vlastnosti vertikálnych a porastových zrážok horských smrečín v TANAP-e. [Diplomová práca] / Zuzana Sitková. – Technická univerzita vo Zvolene. Lesnícka fakulta; Katedra prírodného prostredia. – Školiteľ: Jaroslav Škvarenina. Zvolen: LF TU, 1998.

Práca prezentuje výsledky chemicko-fyzikálnych vlastností zrážkových vôd (pH, elektrická vodivosť, H+, SO42-, NO3-, NH4+, Ca2+, Mg2+, Na+, K+, Al3+) a zaoberá sa výpočtom imisných atmosférických depozícií v poraste a na voľnej ploche za rok 1997. Objektom práce boli porasty v slt Cembreto-Piceetum a Mughetum acidofilum na lokalite Popradské Pleso (1540 m n. m) a v slt Lariceto-Piceetum na lokalite Vyšné Hágy (1140 m n. m). Práca analyzuje a porovnáva chemizmus zrážok na voľnej ploche, podkorunových zrážok a zrážkových vôd stekajúcich po kmeni smreka, smrekovca, limby a kosodreviny.  Na skúmaných plochách bola zistená výrazná acidifikáciu zrážok a látkovo imisné obohatenie porastových zrážok v porovnaní s voľnou plochou. Predpokladá sa, že kyslý imisný vstup prekračujúci kritické hodnoty zohráva významnú úlohu pri rozpade lesných ekosystémov v TANAP-e.

\vspace{0.5cm}
\textbf{Kľúčové slová:} atmosférická depozícia, acidita zrážok, koeficient obohatenia, kritická úroveň, kritická záťaž, porastové zrážky, vertikálne zrážky, elektrická vodivosť, imisie, mokrá depozícia (Katuščák, 2004). \\

\section*{Abstract}

\hspace{1.25cm}In this place, insert text of the abstract including keywords in English or another foreign language. Sem vložte text abstraktu vrátane kľúčových slov v angličtine, prípadne v inom zvolenom cudzom jazyku. 

\vspace{0.5cm}
\textbf{Keywords:} Insert the minimum of 4 keywords that accurately characterize your topic.
