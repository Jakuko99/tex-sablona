\documentclass[12pt]{article}
\usepackage{mathptmx}
\usepackage{tocloft}
\usepackage{tocbasic}
\usepackage{tabularx}
\usepackage{lipsum}
\usepackage{graphicx}
\usepackage{geometry}
\usepackage{float}
\usepackage[framemethod=tikz]{mdframed}
\usepackage{caption}% http://ctan.org/pkg/caption
\captionsetup[table]{justification=raggedleft,singlelinecheck=off}
%\usepackage{hyperref}
%\usepackage[slovak]{babel} % síce opraví ď a ť, ale premenuje niektoré menovky na zlé hodnoty, ktoré ale renewcommand neopraví naspäť

\parskip=5pt plus 1pt % medzera medzi odsekmi
\renewcommand{\baselinestretch}{1.25} % riadkovanie
\geometry{
	a4paper,
	left=3.5cm,
	right=2.5cm,
	top=2.5cm,
	bottom=2.5cm
} % nastavenie okrajov strany

% ----- konfigurácia pomenovania zoznamov a popisov -----
\renewcommand{\cftsecleader}{\cftdotfill{\cftdotsep}} % bodky v obsahu
\renewcommand{\contentsname}{Obsah} % premenovanie zoznamov do slovenčiny
\newcommand\entrynumberwithprefix[2]{#1\enspace#2:\hfill}
\DeclareTOCStyleEntry[ % pridanie obr. pred názov obrázku do zoznamu obrázkov
entrynumberformat=\entrynumberwithprefix{\figurename},
dynnumwidth,
numsep=1em
]{tocline}{figure}
\DeclareTOCStyleEntry[ % pridanie tab. pred názov tabuľky
entrynumberformat=\entrynumberwithprefix{\tablename},
dynnumwidth,
numsep=1em
]{tocline}{table}
\renewcommand{\listfigurename}{Zoznam obrázkov}
\renewcommand{\listtablename}{Zoznam tabuliek}
\renewcommand{\refname}{Zoznam použitej literatúry}
\renewcommand{\figurename}{\textbf{Obr.}} % úprava popisov pre obrázky
\renewcommand{\thefigure}{\textbf{\arabic{figure}}}
%\renewcommand{\thefigure}{\arabic{section}.\arabic{figure}} %štrukturované značenie obrázkov (nefunguje správne)
\renewcommand{\tablename}{\textbf{Tab.}} % zmena popisu tabuľky
\renewcommand{\thetable}{\textbf{\arabic{table}}} % tučné písmo aj pre číslo tabuľky

% ----- vytvorenie zoznamu pre prílohy -----
\newcommand{\listexamplename}{Zoznam príloh}
\newlistof{example}{exp}{\listexamplename}
\newcounter{examplealph}
\newcommand{\example}[1]{%
	\refstepcounter{examplealph}
	\par\noindent\section*{Príloha \Alph{examplealph}: #1}
	\addcontentsline{exp}{example}
	{\textbf{Príloha \protect\numberline{\Alph{examplealph}}}#1}\par}
\renewcommand{\theexamplealph}{\Alph{examplealph}}

\graphicspath{{images/}} % priečinok kde sú uložené obrázky

\begin{document}
\thispagestyle{empty} % skryť číslovanie strany
\pagenumbering{gobble} % vypnutie číslovania strán, obálka práce
\begin{figure}[H] %logo univerzity
	\centering
	\includegraphics[scale=1.2]{logo}
\end{figure}

\begin{center} %titulná strana
	\begin{Large}
		Fakulta elektrotechniky a informačných technológií \\

		\vspace{1.5cm}
		Názov práce \\
		\vspace{0.5cm}
		typ práce \\
		\vspace{2.5cm}
		\textbf{meno priezvisko} \\
		\vfill
	\end{Large}
\end{center}
Vedúci diplomovej práce: vedúci práce \\
Evidenčné číslo: evidenčné číslo práce \\
Žilina 2024 \\

\newpage
\thispagestyle{empty} % skryť číslovanie strany
\pagenumbering{gobble} % vypnutie číslovania strán
\begin{figure}[H] %logo univerzity
	\centering
	\includegraphics[scale=1.2]{logo.jpg}
\end{figure}

\begin{center} %titulná strana
	\begin{Large}
		Fakulta elektrotechniky a informačných technológií \\

		\vspace{2cm}
		Názov práce \\
		\vspace{0.5cm}
		Podnázov práce \\
		\vspace{0.75cm}
		Diplomová práca \\
		\vspace{3cm}
		\textbf{titul, Meno a Priezvisko} \\
		\vfill
	\end{Large}
\end{center}
Študijný program: Názov študijného programu \\
Študijný odbor: názov odboru \\
Školiace pracovisko: Žilinská univerzita v Žiline, \\
Vedúci [Vyberte druh práce]: Titul, meno a priezvisko \\
Konzultant: Titul, meno a priezvisko (ak práca nemá konzultanta, riadok vymažte) \\
Žilina 2024 \\ % vloženie súboru s titulnou stranou, takto je možné pridávať súbory ktoré obsahujú časti práce

\newpage % strana pre zadanie
\thispagestyle{empty}
\begingroup
\begin{LARGE}
	Namiesto tejto strany treba vložiť zadanie záverečnej práce.
	Do elektronickej verzie práce vložte naskenované zadanie záverečnej práce ako obrázok zväčšený na celú veľkosť papiera
\end{LARGE}
\endgroup

\newpage % čestné vyhlásenie
\thispagestyle{empty}
\mbox{}
\vfill
\section*{Čestné vyhlásenie}
Vyhlasujem, že som zadanú diplomovú prácu vypracoval samostatne, pod odborným vedením vedúceho práce/školiteľa a používal som len literatúru uvedenú v práci. \\
\vspace{1cm}
Žilina 3. marca 2022 \\
\vspace{1cm}	
\null\hfill podpis

\newpage % poďakovanie
\thispagestyle{empty}
\mbox{}
\vfill
\section*{Poďakovanie}
(Poďakovanie nie je povinná časť záverečnej práce)

\newpage % abstrakt
\thispagestyle{empty}
\section*{Abstrakt}

\hspace{1.25cm}Abstrakt býva spravidla informatívny a zachováva tematické a štýlové vlastnosti primárneho dokumentu. Podľa možnosti obsahuje kvalitatívnu a kvantitatívnu informáciu obsiahnutú v dokumente. Abstrakt sa píše súvisle ako jeden odsek a jeho rozsah je spravidla 100 až 500 slov. Súčasťou abstraktu je 3 až 5 kľúčových slov (Metodické usmernenie 14/2009-R, 2009; Katuščák, 2004).
Príklad: \\
SITKOVÁ, Zuzana: Hydrochemické vlastnosti vertikálnych a porastových zrážok horských smrečín v TANAP-e. [Diplomová práca] / Zuzana Sitková. – Technická univerzita vo Zvolene. Lesnícka fakulta; Katedra prírodného prostredia. – Školiteľ: Jaroslav Škvarenina. Zvolen: LF TU, 1998.

Práca prezentuje výsledky chemicko-fyzikálnych vlastností zrážkových vôd (pH, elektrická vodivosť, H+, SO42-, NO3-, NH4+, Ca2+, Mg2+, Na+, K+, Al3+) a zaoberá sa výpočtom imisných atmosférických depozícií v poraste a na voľnej ploche za rok 1997. Objektom práce boli porasty v slt Cembreto-Piceetum a Mughetum acidofilum na lokalite Popradské Pleso (1540 m n. m) a v slt Lariceto-Piceetum na lokalite Vyšné Hágy (1140 m n. m). Práca analyzuje a porovnáva chemizmus zrážok na voľnej ploche, podkorunových zrážok a zrážkových vôd stekajúcich po kmeni smreka, smrekovca, limby a kosodreviny.  Na skúmaných plochách bola zistená výrazná acidifikáciu zrážok a látkovo imisné obohatenie porastových zrážok v porovnaní s voľnou plochou. Predpokladá sa, že kyslý imisný vstup prekračujúci kritické hodnoty zohráva významnú úlohu pri rozpade lesných ekosystémov v TANAP-e.

\vspace{0.5cm}
\textbf{Kľúčové slová:} atmosférická depozícia, acidita zrážok, koeficient obohatenia, kritická úroveň, kritická záťaž, porastové zrážky, vertikálne zrážky, elektrická vodivosť, imisie, mokrá depozícia (Katuščák, 2004). \\

\section*{Abstract}

\hspace{1.25cm}In this place, insert text of the abstract including keywords in English or another foreign language. Sem vložte text abstraktu vrátane kľúčových slov v angličtine, prípadne v inom zvolenom cudzom jazyku. 

\vspace{0.5cm}
\textbf{Keywords:} Insert the minimum of 4 keywords that accurately characterize your topic.
 % vloženie úvodných strán

\newpage % obsah, zoznam obrázkov a tabuliek
\thispagestyle{empty}
\tableofcontents
\newpage
\thispagestyle{empty}
\listoffigures
\newpage
\thispagestyle{empty}
\listoftables

\newpage % zoznam skratiek a symbolov
\thispagestyle{empty}
\section*{Zoznam skratiek}

\begin{tabularx}{\textwidth}{ 
		>{\raggedright\arraybackslash}X 
		>{\raggedright\arraybackslash}X
		>{\raggedright\arraybackslash}X }
	\textbf{Skratka} & \textbf{Anglický význam} & \textbf{Slovenský význam} \\
	etc. & et cetera & a tak ďalej
\end{tabularx}

\section*{Zoznam symbolov}

\begin{tabularx}{\textwidth}{ 
		>{\raggedright\arraybackslash}X 
		>{\raggedright\arraybackslash}X
		>{\raggedright\arraybackslash}X }
	\textbf{Symbol} & \textbf{Jednotka} & \textbf{Význam symbolu} \\
	U & Volt (V) & veličina elektrického napätia
\end{tabularx} % vloženie zoznami skratiek a symbolov

% obsahová časť práce začínajúca s úvodom
\newpage
\pagenumbering{arabic}
\setcounter{page}{10} % zapnutie číslovania strán od hodnoty 10
\addcontentsline{toc}{section}{Úvod} % pridanie nečíslovaných kapitol do obsahu
\section*{Úvod} % sekcia bez čísla
\hspace{1.25cm}Postea voluptua quo cu, paulo oportere mea et. Alia equidem id eam, diam oratio phaedrum sit ex. Sea in soluta saperet corrumpit, et sea nullam legendos. Ad abhorreant deseruisse adipiscing per, ei pro idque maiorum mentitum. Has ei porro doctus inimicus. Mea ea postea alterum torquatos, sit cu idque movet reprimique, ex agam indoctum incorrupte mel.
Cum ad rationibus disputando necessitatibus, vel eu bonorum utroque menandri. His an falli democritum intellegam. Maluisset torquatos sit in. Possit mnesarchum efficiendi pro ut. Erat justo molestie ea quo, in vim atqui atomorum abhorreant.~\cite{modernrob}

Práca bola napísaná pomocou použitia \LaTeX, čo je nástroj na písanie profesionálne vyzerajúcich dokumentov pomocou používania príkazov v textovom dokumente, ktoré tvoria s pomocou \TeX \ engine výsledný dokument.

\newpage
\section{Úvod do problematiky}
kapitola 1
\begin{figure}[H] % H - zobrazenie kde sa nachádza v tex súbore
\centering
\includegraphics{logo}
\caption{obrázok}
\label{img:obrazok}
\end{figure}
\subsection{podkapitola}
\begin{figure}[H] % H - zobrazenie kde sa nachádza v tex súbore
\centering
\includegraphics{logo}
\caption{ob}
\label{img:ob}
\end{figure}
\subsubsection{pod-podkapitola}
\begin{table}[H]
	\centering
	\caption{tabulka}
	\label{tab:my-table}
	\begin{tabular}{|l|l|}
\hline
1 & 3 \\ \hline
4 & 2 \\ \hline
	\end{tabular}	
	\end{table}
\section{Teoretické poznatky}
\begin{figure}[H] % H - zobrazenie kde sa nachádza v tex súbore
\centering
\includegraphics{logo}
\caption{test}
\label{img:test}
\end{figure}

% hlavná obsahová časť práce

\newpage % záver
\addcontentsline{toc}{section}{Záver}
\section*{Záver}
text záveru

\newpage % zoznam použitej literatúry
\addcontentsline{toc}{section}{Zoznam použitej literatúry}
\bibliographystyle{ieeetr}
\bibliography{sablona_dp}

\newpage % začiatok príloh
\topskip0pt
\vspace*{\fill}
\addcontentsline{toc}{section}{Prílohy}
\begin{centering} \section*{Prílohy} \end{centering}
\vspace*{\fill}

\newpage % zoznam príloh
\listofexample

\newpage
\example{Príklad prílohy}
\end{document}